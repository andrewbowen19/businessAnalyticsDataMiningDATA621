% Options for packages loaded elsewhere
\PassOptionsToPackage{unicode}{hyperref}
\PassOptionsToPackage{hyphens}{url}
%
\documentclass[
]{article}
\usepackage{amsmath,amssymb}
\usepackage{iftex}
\ifPDFTeX
  \usepackage[T1]{fontenc}
  \usepackage[utf8]{inputenc}
  \usepackage{textcomp} % provide euro and other symbols
\else % if luatex or xetex
  \usepackage{unicode-math} % this also loads fontspec
  \defaultfontfeatures{Scale=MatchLowercase}
  \defaultfontfeatures[\rmfamily]{Ligatures=TeX,Scale=1}
\fi
\usepackage{lmodern}
\ifPDFTeX\else
  % xetex/luatex font selection
\fi
% Use upquote if available, for straight quotes in verbatim environments
\IfFileExists{upquote.sty}{\usepackage{upquote}}{}
\IfFileExists{microtype.sty}{% use microtype if available
  \usepackage[]{microtype}
  \UseMicrotypeSet[protrusion]{basicmath} % disable protrusion for tt fonts
}{}
\makeatletter
\@ifundefined{KOMAClassName}{% if non-KOMA class
  \IfFileExists{parskip.sty}{%
    \usepackage{parskip}
  }{% else
    \setlength{\parindent}{0pt}
    \setlength{\parskip}{6pt plus 2pt minus 1pt}}
}{% if KOMA class
  \KOMAoptions{parskip=half}}
\makeatother
\usepackage{xcolor}
\usepackage[margin=1in]{geometry}
\usepackage{color}
\usepackage{fancyvrb}
\newcommand{\VerbBar}{|}
\newcommand{\VERB}{\Verb[commandchars=\\\{\}]}
\DefineVerbatimEnvironment{Highlighting}{Verbatim}{commandchars=\\\{\}}
% Add ',fontsize=\small' for more characters per line
\usepackage{framed}
\definecolor{shadecolor}{RGB}{248,248,248}
\newenvironment{Shaded}{\begin{snugshade}}{\end{snugshade}}
\newcommand{\AlertTok}[1]{\textcolor[rgb]{0.94,0.16,0.16}{#1}}
\newcommand{\AnnotationTok}[1]{\textcolor[rgb]{0.56,0.35,0.01}{\textbf{\textit{#1}}}}
\newcommand{\AttributeTok}[1]{\textcolor[rgb]{0.13,0.29,0.53}{#1}}
\newcommand{\BaseNTok}[1]{\textcolor[rgb]{0.00,0.00,0.81}{#1}}
\newcommand{\BuiltInTok}[1]{#1}
\newcommand{\CharTok}[1]{\textcolor[rgb]{0.31,0.60,0.02}{#1}}
\newcommand{\CommentTok}[1]{\textcolor[rgb]{0.56,0.35,0.01}{\textit{#1}}}
\newcommand{\CommentVarTok}[1]{\textcolor[rgb]{0.56,0.35,0.01}{\textbf{\textit{#1}}}}
\newcommand{\ConstantTok}[1]{\textcolor[rgb]{0.56,0.35,0.01}{#1}}
\newcommand{\ControlFlowTok}[1]{\textcolor[rgb]{0.13,0.29,0.53}{\textbf{#1}}}
\newcommand{\DataTypeTok}[1]{\textcolor[rgb]{0.13,0.29,0.53}{#1}}
\newcommand{\DecValTok}[1]{\textcolor[rgb]{0.00,0.00,0.81}{#1}}
\newcommand{\DocumentationTok}[1]{\textcolor[rgb]{0.56,0.35,0.01}{\textbf{\textit{#1}}}}
\newcommand{\ErrorTok}[1]{\textcolor[rgb]{0.64,0.00,0.00}{\textbf{#1}}}
\newcommand{\ExtensionTok}[1]{#1}
\newcommand{\FloatTok}[1]{\textcolor[rgb]{0.00,0.00,0.81}{#1}}
\newcommand{\FunctionTok}[1]{\textcolor[rgb]{0.13,0.29,0.53}{\textbf{#1}}}
\newcommand{\ImportTok}[1]{#1}
\newcommand{\InformationTok}[1]{\textcolor[rgb]{0.56,0.35,0.01}{\textbf{\textit{#1}}}}
\newcommand{\KeywordTok}[1]{\textcolor[rgb]{0.13,0.29,0.53}{\textbf{#1}}}
\newcommand{\NormalTok}[1]{#1}
\newcommand{\OperatorTok}[1]{\textcolor[rgb]{0.81,0.36,0.00}{\textbf{#1}}}
\newcommand{\OtherTok}[1]{\textcolor[rgb]{0.56,0.35,0.01}{#1}}
\newcommand{\PreprocessorTok}[1]{\textcolor[rgb]{0.56,0.35,0.01}{\textit{#1}}}
\newcommand{\RegionMarkerTok}[1]{#1}
\newcommand{\SpecialCharTok}[1]{\textcolor[rgb]{0.81,0.36,0.00}{\textbf{#1}}}
\newcommand{\SpecialStringTok}[1]{\textcolor[rgb]{0.31,0.60,0.02}{#1}}
\newcommand{\StringTok}[1]{\textcolor[rgb]{0.31,0.60,0.02}{#1}}
\newcommand{\VariableTok}[1]{\textcolor[rgb]{0.00,0.00,0.00}{#1}}
\newcommand{\VerbatimStringTok}[1]{\textcolor[rgb]{0.31,0.60,0.02}{#1}}
\newcommand{\WarningTok}[1]{\textcolor[rgb]{0.56,0.35,0.01}{\textbf{\textit{#1}}}}
\usepackage{graphicx}
\makeatletter
\def\maxwidth{\ifdim\Gin@nat@width>\linewidth\linewidth\else\Gin@nat@width\fi}
\def\maxheight{\ifdim\Gin@nat@height>\textheight\textheight\else\Gin@nat@height\fi}
\makeatother
% Scale images if necessary, so that they will not overflow the page
% margins by default, and it is still possible to overwrite the defaults
% using explicit options in \includegraphics[width, height, ...]{}
\setkeys{Gin}{width=\maxwidth,height=\maxheight,keepaspectratio}
% Set default figure placement to htbp
\makeatletter
\def\fps@figure{htbp}
\makeatother
\setlength{\emergencystretch}{3em} % prevent overfull lines
\providecommand{\tightlist}{%
  \setlength{\itemsep}{0pt}\setlength{\parskip}{0pt}}
\setcounter{secnumdepth}{-\maxdimen} % remove section numbering
\ifLuaTeX
  \usepackage{selnolig}  % disable illegal ligatures
\fi
\IfFileExists{bookmark.sty}{\usepackage{bookmark}}{\usepackage{hyperref}}
\IfFileExists{xurl.sty}{\usepackage{xurl}}{} % add URL line breaks if available
\urlstyle{same}
\hypersetup{
  pdftitle={Data621: HW2},
  pdfauthor={Shoshanna Farber, Josh Forster, Glen Davis, Andrew Bowen, Charles Ugiagbe},
  hidelinks,
  pdfcreator={LaTeX via pandoc}}

\title{Data621: HW2}
\author{Shoshanna Farber, Josh Forster, Glen Davis, Andrew Bowen,
Charles Ugiagbe}
\date{2023-10-05}

\begin{document}
\maketitle

\begin{Shaded}
\begin{Highlighting}[]
\FunctionTok{library}\NormalTok{(tidyverse)}
\end{Highlighting}
\end{Shaded}

\begin{verbatim}
## -- Attaching core tidyverse packages ------------------------ tidyverse 2.0.0 --
## v dplyr     1.1.2     v readr     2.1.4
## v forcats   1.0.0     v stringr   1.5.0
## v ggplot2   3.4.2     v tibble    3.2.1
## v lubridate 1.9.2     v tidyr     1.3.0
## v purrr     1.0.1     
## -- Conflicts ------------------------------------------ tidyverse_conflicts() --
## x dplyr::filter() masks stats::filter()
## x dplyr::lag()    masks stats::lag()
## i Use the ]8;;http://conflicted.r-lib.org/conflicted package]8;; to force all conflicts to become errors
\end{verbatim}

\begin{Shaded}
\begin{Highlighting}[]
\FunctionTok{library}\NormalTok{(geomtextpath)}
\end{Highlighting}
\end{Shaded}

\hypertarget{overview}{%
\subsection{Overview:}\label{overview}}

In this homework assignment, you will work through various
classification metrics. You will be asked to create functions in R to
carry out the various calculations. You will also investigate some
functions in packages that will let you obtain the equivalent results.
Finally, you will create graphical output that also can be used to
evaluate the output of classification models, such as binary logistic
regression.

\hypertarget{supplemental-material}{%
\subsection{Supplemental Material:}\label{supplemental-material}}

\begin{itemize}
\item
  Applied Predictive Modeling, Ch. 11 (provided as a PDF file).
\item
  Web tutorials: \url{http://www.saedsayad.com/model_evaluation_c.htm}
\end{itemize}

\hypertarget{deliverables-100-points}{%
\subsection{Deliverables (100 Points):}\label{deliverables-100-points}}

\begin{itemize}
\tightlist
\item
  Upon following the instructions below, use your created R functions
  and the other packages to generate the classification metrics for the
  provided data set. A write-up of your solutions submitted in PDF
  format.
\end{itemize}

\hypertarget{instructions}{%
\subsection{Instructions:}\label{instructions}}

Complete each of the following steps as instructed:

\begin{enumerate}
\def\labelenumi{\arabic{enumi}.}
\tightlist
\item
  Download the classification output data set (attached in Blackboard to
  the assignment).
\end{enumerate}

\begin{Shaded}
\begin{Highlighting}[]
\NormalTok{class\_df }\OtherTok{\textless{}{-}} \FunctionTok{as.data.frame}\NormalTok{(}\FunctionTok{read.csv}\NormalTok{(}\StringTok{\textquotesingle{}https://raw.githubusercontent.com/andrewbowen19/businessAnalyticsDataMiningDATA621/main/data/hw2\_input\_classification{-}output{-}data.csv\textquotesingle{}}\NormalTok{))}
\FunctionTok{head}\NormalTok{(class\_df)}
\end{Highlighting}
\end{Shaded}

\begin{verbatim}
##   pregnant glucose diastolic skinfold insulin  bmi pedigree age class
## 1        7     124        70       33     215 25.5    0.161  37     0
## 2        2     122        76       27     200 35.9    0.483  26     0
## 3        3     107        62       13      48 22.9    0.678  23     1
## 4        1      91        64       24       0 29.2    0.192  21     0
## 5        4      83        86       19       0 29.3    0.317  34     0
## 6        1     100        74       12      46 19.5    0.149  28     0
##   scored.class scored.probability
## 1            0         0.32845226
## 2            0         0.27319044
## 3            0         0.10966039
## 4            0         0.05599835
## 5            0         0.10049072
## 6            0         0.05515460
\end{verbatim}

The dataset provided includes several attributes to predict whether or
not a patient has diabetes.

\begin{enumerate}
\def\labelenumi{\arabic{enumi}.}
\setcounter{enumi}{1}
\item
  The data set has three key columns we will use:

  \begin{enumerate}
  \def\labelenumii{\alph{enumii}.}
  \item
    class: the actual class for the observation
  \item
    scored.class: the predicted class for the observation (based on a
    threshold of \(0.5\))
  \item
    scored.probability: the predicted probability of success for the
    observation
  \end{enumerate}
\end{enumerate}

Use the table() function to get the raw confusion matrix for this scored
dataset. Make sure you understand the output. In particular, do the rows
represent the actual or predicted class? The columns?

\begin{Shaded}
\begin{Highlighting}[]
\NormalTok{class\_df }\OtherTok{\textless{}{-}}\NormalTok{ class\_df }\SpecialCharTok{|\textgreater{}}
  \FunctionTok{mutate}\NormalTok{(}\AttributeTok{class =} \FunctionTok{factor}\NormalTok{(class, }\AttributeTok{levels=}\FunctionTok{c}\NormalTok{(}\DecValTok{1}\NormalTok{,}\DecValTok{0}\NormalTok{)),}
         \AttributeTok{scored.class =} \FunctionTok{factor}\NormalTok{(scored.class, }\AttributeTok{levels=}\FunctionTok{c}\NormalTok{(}\DecValTok{1}\NormalTok{,}\DecValTok{0}\NormalTok{)))}
\NormalTok{keep }\OtherTok{\textless{}{-}} \FunctionTok{c}\NormalTok{(}\StringTok{"scored.class"}\NormalTok{, }\StringTok{"class"}\NormalTok{)}
\NormalTok{class\_subset }\OtherTok{\textless{}{-}}\NormalTok{ class\_df }\SpecialCharTok{|\textgreater{}}
    \FunctionTok{select}\NormalTok{(}\FunctionTok{all\_of}\NormalTok{(keep))}
\NormalTok{confusion\_matrix }\OtherTok{\textless{}{-}} \FunctionTok{table}\NormalTok{(class\_subset)}
\NormalTok{confusion\_matrix}
\end{Highlighting}
\end{Shaded}

\begin{verbatim}
##             class
## scored.class   1   0
##            1  27   5
##            0  30 119
\end{verbatim}

The row labels in the confusion matrix represent the predicted class,
while the column labels represent the actual class. The values in the
confusion matrix correspond to the number of observations
correctly/incorrectly predicted for this dataframe. There we 27
instances where a patient was accurately diagnosed with diabetes (true
positive) and 119 instances where a patient without diabetes was
classified as not having diabetes (true negative). There were 30
instances where a patient with diabetes was incorrectly classified as
not having diabetes (false negative) and 5 instances of a patient being
incorrectly diagnosed with diabetes (false positive).

\begin{enumerate}
\def\labelenumi{\arabic{enumi}.}
\setcounter{enumi}{2}
\tightlist
\item
  Write a function that takes the data set as a dataframe, with actual
  and predicted classifications identified, and returns the accuracy of
  the predictions.
\end{enumerate}

\[Accuracy = \frac{TP + TN}{TP + FP + TN + FN}\]

\begin{Shaded}
\begin{Highlighting}[]
\NormalTok{accuracy.func }\OtherTok{\textless{}{-}} \ControlFlowTok{function}\NormalTok{(input\_df) \{}
\NormalTok{    accuracy }\OtherTok{\textless{}{-}}\NormalTok{ input\_df }\SpecialCharTok{|\textgreater{}}\NormalTok{ dplyr}\SpecialCharTok{::}\FunctionTok{mutate}\NormalTok{(}\AttributeTok{correct=}\FunctionTok{ifelse}\NormalTok{(class}\SpecialCharTok{==}\NormalTok{scored.class,}\DecValTok{1}\NormalTok{,}\DecValTok{0}\NormalTok{)) }\SpecialCharTok{|\textgreater{}} \FunctionTok{summarise}\NormalTok{(}\AttributeTok{total\_correct =} \FunctionTok{sum}\NormalTok{(correct),}\AttributeTok{accuracy=}\NormalTok{total\_correct}\SpecialCharTok{/}\FunctionTok{n}\NormalTok{()) }\SpecialCharTok{|\textgreater{}} \FunctionTok{select}\NormalTok{(}\FunctionTok{c}\NormalTok{(accuracy))}
\NormalTok{\}}

\NormalTok{acc\_val }\OtherTok{\textless{}{-}} \FunctionTok{accuracy.func}\NormalTok{(class\_subset)}
\FunctionTok{print}\NormalTok{(}\FunctionTok{round}\NormalTok{(acc\_val, }\DecValTok{2}\NormalTok{))}
\end{Highlighting}
\end{Shaded}

\begin{verbatim}
##   accuracy
## 1     0.81
\end{verbatim}

\begin{enumerate}
\def\labelenumi{\arabic{enumi}.}
\setcounter{enumi}{3}
\tightlist
\item
  Write a function that takes the data set as a dataframe, with actual
  and predicted classifications identified, and returns the
  classification error rate of the predictions.
\end{enumerate}

\[Classification\ Error\ Rate = \frac{FP + FN}{TP + FP + TN + FN}\]

\begin{Shaded}
\begin{Highlighting}[]
\NormalTok{error.func }\OtherTok{\textless{}{-}} \ControlFlowTok{function}\NormalTok{(input\_df) \{}
\NormalTok{    accuracy }\OtherTok{\textless{}{-}}\NormalTok{ input\_df }\SpecialCharTok{|\textgreater{}}\NormalTok{ dplyr}\SpecialCharTok{::}\FunctionTok{mutate}\NormalTok{(}\AttributeTok{incorrect=}\FunctionTok{ifelse}\NormalTok{(class}\SpecialCharTok{==}\NormalTok{scored.class,}\DecValTok{0}\NormalTok{,}\DecValTok{1}\NormalTok{)) }\SpecialCharTok{|\textgreater{}} \FunctionTok{summarise}\NormalTok{(}\AttributeTok{total\_incorrect =} \FunctionTok{sum}\NormalTok{(incorrect),}\AttributeTok{error\_rate=}\NormalTok{total\_incorrect}\SpecialCharTok{/}\FunctionTok{n}\NormalTok{()) }\SpecialCharTok{|\textgreater{}} \FunctionTok{select}\NormalTok{(}\FunctionTok{c}\NormalTok{(error\_rate))}
\NormalTok{\}}

\NormalTok{error\_val }\OtherTok{\textless{}{-}} \FunctionTok{error.func}\NormalTok{(class\_subset)}
\FunctionTok{print}\NormalTok{(}\FunctionTok{round}\NormalTok{(error\_val, }\DecValTok{2}\NormalTok{))}
\end{Highlighting}
\end{Shaded}

\begin{verbatim}
##   error_rate
## 1       0.19
\end{verbatim}

Verify that you get an accuracy and an error rate that sums to one.

\begin{Shaded}
\begin{Highlighting}[]
\NormalTok{(}\FunctionTok{as.numeric}\NormalTok{(acc\_val) }\SpecialCharTok{+} \FunctionTok{as.numeric}\NormalTok{(error\_val))}
\end{Highlighting}
\end{Shaded}

\begin{verbatim}
## [1] 1
\end{verbatim}

Confirmed. The accuracy and error rate values do add up to one.

\begin{enumerate}
\def\labelenumi{\arabic{enumi}.}
\setcounter{enumi}{4}
\tightlist
\item
  Write a function that takes the data set as a dataframe, with actual
  and predicted classifications identified, and returns the precision of
  the predictions.
\end{enumerate}

\[Precision = \frac{TP}{TP + FP}\]

\begin{Shaded}
\begin{Highlighting}[]
\NormalTok{precision.func }\OtherTok{\textless{}{-}} \ControlFlowTok{function}\NormalTok{(input\_df) \{}
\NormalTok{    precision }\OtherTok{\textless{}{-}}\NormalTok{ input\_df }\SpecialCharTok{|\textgreater{}}\NormalTok{ dplyr}\SpecialCharTok{::}\FunctionTok{filter}\NormalTok{(scored.class}\SpecialCharTok{==}\DecValTok{1}\NormalTok{) }\SpecialCharTok{|\textgreater{}} \FunctionTok{mutate}\NormalTok{(}\AttributeTok{prec=}\FunctionTok{ifelse}\NormalTok{(class}\SpecialCharTok{==}\NormalTok{scored.class,}\DecValTok{1}\NormalTok{,}\DecValTok{0}\NormalTok{)) }\SpecialCharTok{|\textgreater{}} 
        \FunctionTok{summarise}\NormalTok{(}\AttributeTok{total\_prec =} \FunctionTok{sum}\NormalTok{(prec),}\AttributeTok{precision\_rate=}\NormalTok{total\_prec}\SpecialCharTok{/}\FunctionTok{n}\NormalTok{()) }\SpecialCharTok{|\textgreater{}} \FunctionTok{select}\NormalTok{(}\FunctionTok{c}\NormalTok{(precision\_rate))}
\NormalTok{\}}

\NormalTok{prec\_val }\OtherTok{\textless{}{-}} \FunctionTok{precision.func}\NormalTok{(class\_subset)}
\FunctionTok{print}\NormalTok{(}\FunctionTok{round}\NormalTok{(prec\_val, }\DecValTok{2}\NormalTok{))}
\end{Highlighting}
\end{Shaded}

\begin{verbatim}
##   precision_rate
## 1           0.84
\end{verbatim}

\begin{enumerate}
\def\labelenumi{\arabic{enumi}.}
\setcounter{enumi}{5}
\tightlist
\item
  Write a function that takes the data set as a dataframe, with actual
  and predicted classifications identified, and returns the sensitivity
  of the predictions. Sensitivity is also known as recall.
\end{enumerate}

\[Sensitivity = \frac{TP}{TP + FN}\]

\begin{Shaded}
\begin{Highlighting}[]
\NormalTok{sensitivity.func }\OtherTok{\textless{}{-}} \ControlFlowTok{function}\NormalTok{(input\_df) \{}
\NormalTok{    sensitivity }\OtherTok{\textless{}{-}}\NormalTok{ input\_df }\SpecialCharTok{|\textgreater{}}\NormalTok{ dplyr}\SpecialCharTok{::}\FunctionTok{filter}\NormalTok{(class}\SpecialCharTok{==}\DecValTok{1}\NormalTok{) }\SpecialCharTok{|\textgreater{}} \FunctionTok{mutate}\NormalTok{(}\AttributeTok{sens=}\FunctionTok{ifelse}\NormalTok{(class}\SpecialCharTok{==}\NormalTok{scored.class,}\DecValTok{1}\NormalTok{,}\DecValTok{0}\NormalTok{)) }\SpecialCharTok{|\textgreater{}} 
        \FunctionTok{summarise}\NormalTok{(}\AttributeTok{total\_sens =} \FunctionTok{sum}\NormalTok{(sens),}\AttributeTok{sensitivity\_rate=}\NormalTok{total\_sens}\SpecialCharTok{/}\FunctionTok{n}\NormalTok{()) }\SpecialCharTok{|\textgreater{}} \FunctionTok{select}\NormalTok{(}\FunctionTok{c}\NormalTok{(sensitivity\_rate))}
\NormalTok{\}}

\NormalTok{sens\_val }\OtherTok{\textless{}{-}} \FunctionTok{sensitivity.func}\NormalTok{(class\_subset)}
\FunctionTok{print}\NormalTok{(}\FunctionTok{round}\NormalTok{(sens\_val, }\DecValTok{2}\NormalTok{))}
\end{Highlighting}
\end{Shaded}

\begin{verbatim}
##   sensitivity_rate
## 1             0.47
\end{verbatim}

\begin{enumerate}
\def\labelenumi{\arabic{enumi}.}
\setcounter{enumi}{6}
\tightlist
\item
  Write a function that takes the data set as a dataframe, with actual
  and predicted classifications identified, and returns the specificity
  of the predictions.
\end{enumerate}

\[Specificity = \frac{TN}{TN + FP}\]

\begin{Shaded}
\begin{Highlighting}[]
\NormalTok{specificity.func }\OtherTok{\textless{}{-}} \ControlFlowTok{function}\NormalTok{(input\_df) \{}
\NormalTok{    specificity }\OtherTok{\textless{}{-}}\NormalTok{ input\_df }\SpecialCharTok{|\textgreater{}}\NormalTok{ dplyr}\SpecialCharTok{::}\FunctionTok{filter}\NormalTok{(class}\SpecialCharTok{==}\DecValTok{0}\NormalTok{) }\SpecialCharTok{|\textgreater{}} \FunctionTok{mutate}\NormalTok{(}\AttributeTok{spec=}\FunctionTok{ifelse}\NormalTok{(class}\SpecialCharTok{==}\NormalTok{scored.class,}\DecValTok{1}\NormalTok{,}\DecValTok{0}\NormalTok{)) }\SpecialCharTok{|\textgreater{}} 
        \FunctionTok{summarise}\NormalTok{(}\AttributeTok{total\_spec =} \FunctionTok{sum}\NormalTok{(spec),}\AttributeTok{specificity\_rate=}\NormalTok{total\_spec}\SpecialCharTok{/}\FunctionTok{n}\NormalTok{()) }\SpecialCharTok{|\textgreater{}} \FunctionTok{select}\NormalTok{(}\FunctionTok{c}\NormalTok{(specificity\_rate))}
\NormalTok{\}}

\NormalTok{spec\_val }\OtherTok{\textless{}{-}} \FunctionTok{specificity.func}\NormalTok{(class\_subset)}
\FunctionTok{print}\NormalTok{(}\FunctionTok{round}\NormalTok{(spec\_val, }\DecValTok{2}\NormalTok{))}
\end{Highlighting}
\end{Shaded}

\begin{verbatim}
##   specificity_rate
## 1             0.96
\end{verbatim}

\begin{enumerate}
\def\labelenumi{\arabic{enumi}.}
\setcounter{enumi}{7}
\tightlist
\item
  Write a function that takes the data set as a dataframe, with actual
  and predicted classifications identified, and returns the F1 score of
  the predictions.
\end{enumerate}

\[F1\ Score = \frac{2 \times Precision \times Sensitivity}{Precision + Sensitivity}\]

\begin{Shaded}
\begin{Highlighting}[]
\NormalTok{f1.score }\OtherTok{\textless{}{-}} \ControlFlowTok{function}\NormalTok{(input\_df) \{}
\NormalTok{    f1}\OtherTok{\textless{}{-}}\NormalTok{(}\DecValTok{2}\SpecialCharTok{*}\FunctionTok{precision.func}\NormalTok{(input\_df)}\SpecialCharTok{*}\FunctionTok{sensitivity.func}\NormalTok{(input\_df))}\SpecialCharTok{/}\NormalTok{(}\FunctionTok{precision.func}\NormalTok{(input\_df)}\SpecialCharTok{+}\FunctionTok{sensitivity.func}\NormalTok{(input\_df))}
\NormalTok{\}}

\NormalTok{f1\_val }\OtherTok{\textless{}{-}} \FunctionTok{f1.score}\NormalTok{(class\_subset)}
\FunctionTok{colnames}\NormalTok{(f1\_val) }\OtherTok{\textless{}{-}} \StringTok{"f1\_score"}
\FunctionTok{print}\NormalTok{(}\FunctionTok{round}\NormalTok{(f1\_val, }\DecValTok{2}\NormalTok{))}
\end{Highlighting}
\end{Shaded}

\begin{verbatim}
##   f1_score
## 1     0.61
\end{verbatim}

\begin{enumerate}
\def\labelenumi{\arabic{enumi}.}
\setcounter{enumi}{8}
\tightlist
\item
  Before we move on, let's consider a question that was asked: What are
  the bounds on the F1 score? Show that the F1 score will always be
  between \(0\) and \(1\). (Hint: If \(0 < a < 1\) and \(0 < b < 1\)
  then \(ab < a\).)
\end{enumerate}

Result of the F1 function given that the maximum value for precision and
sensitivity is 1 if every single value were correctly predicted:

\[\frac{2 * 1 * 1}{1+1} = \frac{2}{2} = 1\]

Alternatively, the worst case scenario for the metrics from a
classification model would if every single prediction was incorrect:

\[\frac{2 * 0 * 0}{0+0} = \frac{0}{0}\]

Let's show graphically that even if one of the scores was
perfect/imperfect the maximum value as assumed before would be 1

\begin{Shaded}
\begin{Highlighting}[]
\NormalTok{x }\OtherTok{\textless{}{-}} \FunctionTok{seq}\NormalTok{(}\DecValTok{0}\NormalTok{, }\DecValTok{1}\NormalTok{, }\AttributeTok{by=}\FloatTok{0.01}\NormalTok{)}

\CommentTok{\# Calculate the function values}
\NormalTok{y }\OtherTok{\textless{}{-}}\NormalTok{ (}\DecValTok{2}\SpecialCharTok{*}\NormalTok{ x }\SpecialCharTok{*} \DecValTok{1}\NormalTok{)}\SpecialCharTok{/}\NormalTok{(x}\SpecialCharTok{+}\DecValTok{1}\NormalTok{)}

\CommentTok{\# Create a data frame with x and y values}
\NormalTok{df }\OtherTok{\textless{}{-}} \FunctionTok{data.frame}\NormalTok{(}\AttributeTok{x =}\NormalTok{ x, }\AttributeTok{y =}\NormalTok{ y)}

\FunctionTok{ggplot}\NormalTok{(}\AttributeTok{data =}\NormalTok{ df, }\FunctionTok{aes}\NormalTok{(}\AttributeTok{x =}\NormalTok{ x, }\AttributeTok{y =}\NormalTok{ y)) }\SpecialCharTok{+}
  \FunctionTok{geom\_point}\NormalTok{() }\SpecialCharTok{+}
  \FunctionTok{labs}\NormalTok{(}\AttributeTok{x =} \StringTok{"x"}\NormalTok{, }\AttributeTok{y =} \StringTok{"f(x)"}\NormalTok{,}\AttributeTok{title =} \StringTok{"F1 Score between 0 and 1"}\NormalTok{) }\SpecialCharTok{+}
  \FunctionTok{theme\_light}\NormalTok{()}
\end{Highlighting}
\end{Shaded}

\includegraphics{homeworks2_metrics_files/figure-latex/unnamed-chunk-11-1.pdf}

\begin{enumerate}
\def\labelenumi{\arabic{enumi}.}
\setcounter{enumi}{9}
\tightlist
\item
  Write a function that generates an ROC curve from a data set with a
  true classification column (class in our example) and a probability
  column (scored.probability in our example). Your function should
  return a list that includes the plot of the ROC curve and a vector
  that contains the calculated area under the curve (AUC). Note that I
  recommend using a sequence of thresholds ranging from 0 to 1 at 0.01
  intervals.
\end{enumerate}

\begin{Shaded}
\begin{Highlighting}[]
\NormalTok{keep }\OtherTok{\textless{}{-}} \FunctionTok{c}\NormalTok{(}\StringTok{"class"}\NormalTok{, }\StringTok{"scored.probability"}\NormalTok{)}
\NormalTok{class\_subset }\OtherTok{\textless{}{-}}\NormalTok{ class\_df }\SpecialCharTok{|\textgreater{}}
    \FunctionTok{select}\NormalTok{(}\FunctionTok{all\_of}\NormalTok{(keep))}
\NormalTok{roc.func }\OtherTok{\textless{}{-}} \ControlFlowTok{function}\NormalTok{(input\_df)\{}
\NormalTok{    plot\_points\_df }\OtherTok{\textless{}{-}} \FunctionTok{as.data.frame}\NormalTok{(}\FunctionTok{matrix}\NormalTok{(}\AttributeTok{ncol =} \DecValTok{3}\NormalTok{, }\AttributeTok{nrow =} \DecValTok{0}\NormalTok{))}
\NormalTok{    cols }\OtherTok{\textless{}{-}} \FunctionTok{c}\NormalTok{(}\StringTok{"threshold"}\NormalTok{, }\StringTok{"fpr"}\NormalTok{, }\StringTok{"tpr"}\NormalTok{)}
    \FunctionTok{colnames}\NormalTok{(plot\_points\_df) }\OtherTok{\textless{}{-}}\NormalTok{ cols}
\NormalTok{    thresholds }\OtherTok{\textless{}{-}} \FunctionTok{seq}\NormalTok{(}\AttributeTok{from =} \DecValTok{0}\NormalTok{, }\AttributeTok{to =} \DecValTok{1}\NormalTok{, }\AttributeTok{by =} \FloatTok{0.01}\NormalTok{)}
\NormalTok{    copy }\OtherTok{\textless{}{-}}\NormalTok{ input\_df}
    \ControlFlowTok{for}\NormalTok{ (i }\ControlFlowTok{in} \DecValTok{1}\SpecialCharTok{:}\FunctionTok{length}\NormalTok{(thresholds))\{}
\NormalTok{        threshold }\OtherTok{\textless{}{-}}\NormalTok{ thresholds[i]}
\NormalTok{        copy }\OtherTok{\textless{}{-}}\NormalTok{ copy }\SpecialCharTok{|\textgreater{}}
            \FunctionTok{mutate}\NormalTok{(}\AttributeTok{scored.class =} \FunctionTok{ifelse}\NormalTok{(scored.probability }\SpecialCharTok{\textgreater{}}\NormalTok{ threshold, }\DecValTok{1}\NormalTok{, }\DecValTok{0}\NormalTok{))}
\NormalTok{        fpr }\OtherTok{\textless{}{-}} \DecValTok{1} \SpecialCharTok{{-}} \FunctionTok{as.numeric}\NormalTok{(}\FunctionTok{specificity.func}\NormalTok{(copy))}
\NormalTok{        tpr }\OtherTok{\textless{}{-}} \FunctionTok{as.numeric}\NormalTok{(}\FunctionTok{sensitivity.func}\NormalTok{(copy))}
\NormalTok{        new\_row }\OtherTok{\textless{}{-}} \FunctionTok{as.data.frame}\NormalTok{(}\FunctionTok{t}\NormalTok{(}\FunctionTok{c}\NormalTok{(threshold, fpr, tpr)))}
        \FunctionTok{colnames}\NormalTok{(new\_row) }\OtherTok{\textless{}{-}}\NormalTok{ cols}
\NormalTok{        plot\_points\_df }\OtherTok{\textless{}{-}} \FunctionTok{rbind}\NormalTok{(plot\_points\_df, new\_row)}
\NormalTok{    \}}
\NormalTok{    roc\_plot }\OtherTok{\textless{}{-}} \FunctionTok{ggplot}\NormalTok{(plot\_points\_df, }\FunctionTok{aes}\NormalTok{(}\AttributeTok{x =}\NormalTok{ fpr, }\AttributeTok{y =}\NormalTok{ tpr)) }\SpecialCharTok{+} 
        \FunctionTok{geom\_point}\NormalTok{() }\SpecialCharTok{+}
        \FunctionTok{geom\_line}\NormalTok{() }\SpecialCharTok{+}
        \FunctionTok{geom\_labelabline}\NormalTok{(}\AttributeTok{intercept =} \DecValTok{0}\NormalTok{, }\AttributeTok{slope =} \DecValTok{1}\NormalTok{, }\AttributeTok{label =} \StringTok{"Random Classifier"}\NormalTok{,}
                        \AttributeTok{linetype =} \StringTok{"dashed"}\NormalTok{, }\AttributeTok{color =} \StringTok{"red"}\NormalTok{) }\SpecialCharTok{+}
        \FunctionTok{labs}\NormalTok{(}\AttributeTok{x =} \StringTok{"False Positive Rate (1 {-} Specifity)"}\NormalTok{,}
             \AttributeTok{y =} \StringTok{"True Positive Rate (Sensitivity)"}\NormalTok{,}
             \AttributeTok{title =} \StringTok{"ROC Curve"}\NormalTok{) }\SpecialCharTok{+}
        \FunctionTok{theme\_light}\NormalTok{()}
\NormalTok{    auc }\OtherTok{\textless{}{-}} \DecValTok{0}
\NormalTok{    plot\_points\_df }\OtherTok{\textless{}{-}}\NormalTok{ plot\_points\_df }\SpecialCharTok{|\textgreater{}}
        \FunctionTok{arrange}\NormalTok{(fpr)}
    \ControlFlowTok{for}\NormalTok{ (k }\ControlFlowTok{in} \DecValTok{2}\SpecialCharTok{:}\FunctionTok{nrow}\NormalTok{(plot\_points\_df))\{}
\NormalTok{        x2 }\OtherTok{\textless{}{-}}\NormalTok{ plot\_points\_df[k, }\DecValTok{2}\NormalTok{]}
\NormalTok{        x1 }\OtherTok{\textless{}{-}}\NormalTok{ plot\_points\_df[k }\SpecialCharTok{{-}} \DecValTok{1}\NormalTok{, }\DecValTok{2}\NormalTok{]}
\NormalTok{        y2 }\OtherTok{\textless{}{-}}\NormalTok{ plot\_points\_df[k, }\DecValTok{3}\NormalTok{]}
\NormalTok{        y1 }\OtherTok{\textless{}{-}}\NormalTok{ plot\_points\_df[k }\SpecialCharTok{{-}} \DecValTok{1}\NormalTok{, }\DecValTok{3}\NormalTok{]}
\NormalTok{        trap\_area }\OtherTok{\textless{}{-}}\NormalTok{ (x2 }\SpecialCharTok{{-}}\NormalTok{ x1) }\SpecialCharTok{*}\NormalTok{ ((y2 }\SpecialCharTok{+}\NormalTok{ y1)}\SpecialCharTok{/}\DecValTok{2}\NormalTok{)}
\NormalTok{        auc }\OtherTok{\textless{}{-}}\NormalTok{ auc }\SpecialCharTok{+}\NormalTok{ trap\_area}
\NormalTok{    \}}
\NormalTok{    roc\_plot\_and\_auc }\OtherTok{\textless{}{-}} \FunctionTok{list}\NormalTok{(}\AttributeTok{ROC\_Plot =}\NormalTok{ roc\_plot, }\AttributeTok{AUC =} \FunctionTok{round}\NormalTok{(auc, }\DecValTok{2}\NormalTok{))}
    \FunctionTok{return}\NormalTok{(roc\_plot\_and\_auc)}
\NormalTok{\}}

\NormalTok{roc\_plot\_and\_auc }\OtherTok{\textless{}{-}} \FunctionTok{roc.func}\NormalTok{(class\_subset)}
\NormalTok{roc\_plot\_and\_auc}
\end{Highlighting}
\end{Shaded}

\begin{verbatim}
## $ROC_Plot
\end{verbatim}

\includegraphics{homeworks2_metrics_files/figure-latex/unnamed-chunk-12-1.pdf}

\begin{verbatim}
## 
## $AUC
## [1] 0.85
\end{verbatim}

\begin{enumerate}
\def\labelenumi{\arabic{enumi}.}
\setcounter{enumi}{10}
\tightlist
\item
  Use your created R functions and the provided classification output
  data set to produce all of the classification metrics discussed above.
\end{enumerate}

\begin{Shaded}
\begin{Highlighting}[]
\NormalTok{metrics\_table }\OtherTok{\textless{}{-}} \FunctionTok{cbind}\NormalTok{(acc\_val, error\_val, prec\_val, sens\_val, spec\_val, f1\_val) }\SpecialCharTok{|\textgreater{}}
  \FunctionTok{pivot\_longer}\NormalTok{(accuracy}\SpecialCharTok{:}\NormalTok{f1\_score,}
               \AttributeTok{names\_to=}\StringTok{"metric"}\NormalTok{,}
               \AttributeTok{values\_to=}\StringTok{"value"}\NormalTok{)}

\CommentTok{\#knitr::kable(metrics\_table)}
\end{Highlighting}
\end{Shaded}

\begin{enumerate}
\def\labelenumi{\arabic{enumi}.}
\setcounter{enumi}{11}
\tightlist
\item
  Investigate the caret package. In particular, consider the functions
  confusionMatrix, sensitivity, and specificity. Apply the functions to
  the data set. How do the results compare with your own functions?
\end{enumerate}

\begin{Shaded}
\begin{Highlighting}[]
\CommentTok{\# load package}
\FunctionTok{library}\NormalTok{(caret)}
\end{Highlighting}
\end{Shaded}

\begin{verbatim}
## Loading required package: lattice
\end{verbatim}

\begin{verbatim}
## 
## Attaching package: 'caret'
\end{verbatim}

\begin{verbatim}
## The following object is masked from 'package:purrr':
## 
##     lift
\end{verbatim}

\begin{Shaded}
\begin{Highlighting}[]
\CommentTok{\# confusionMatrix}
\NormalTok{(conf\_matrix }\OtherTok{\textless{}{-}} \FunctionTok{confusionMatrix}\NormalTok{(class\_df}\SpecialCharTok{$}\NormalTok{scored.class,}
                \AttributeTok{reference =}\NormalTok{ class\_df}\SpecialCharTok{$}\NormalTok{class,}
                \AttributeTok{positive =} \StringTok{\textquotesingle{}1\textquotesingle{}}\NormalTok{))}
\end{Highlighting}
\end{Shaded}

\begin{verbatim}
## Confusion Matrix and Statistics
## 
##           Reference
## Prediction   1   0
##          1  27   5
##          0  30 119
##                                           
##                Accuracy : 0.8066          
##                  95% CI : (0.7415, 0.8615)
##     No Information Rate : 0.6851          
##     P-Value [Acc > NIR] : 0.0001712       
##                                           
##                   Kappa : 0.4916          
##                                           
##  Mcnemar's Test P-Value : 4.976e-05       
##                                           
##             Sensitivity : 0.4737          
##             Specificity : 0.9597          
##          Pos Pred Value : 0.8438          
##          Neg Pred Value : 0.7987          
##              Prevalence : 0.3149          
##          Detection Rate : 0.1492          
##    Detection Prevalence : 0.1768          
##       Balanced Accuracy : 0.7167          
##                                           
##        'Positive' Class : 1               
## 
\end{verbatim}

\begin{Shaded}
\begin{Highlighting}[]
\CommentTok{\# sensitivity}
\NormalTok{caret\_sens }\OtherTok{\textless{}{-}} \FunctionTok{sensitivity}\NormalTok{(conf\_matrix}\SpecialCharTok{$}\NormalTok{table)}

\CommentTok{\# specificity}
\NormalTok{caret\_spec }\OtherTok{\textless{}{-}} \FunctionTok{specificity}\NormalTok{(conf\_matrix}\SpecialCharTok{$}\NormalTok{table)}

\CommentTok{\# compare functions}
\NormalTok{caret\_calcs }\OtherTok{\textless{}{-}} \FunctionTok{as.data.frame}\NormalTok{(}\FunctionTok{cbind}\NormalTok{(caret\_sens, caret\_spec)) }
\FunctionTok{colnames}\NormalTok{(caret\_calcs) }\OtherTok{\textless{}{-}} \FunctionTok{c}\NormalTok{(}\StringTok{"sens\_val"}\NormalTok{, }\StringTok{"spec\_val"}\NormalTok{)}
\NormalTok{caret\_calcs }\SpecialCharTok{|\textgreater{}}
  \FunctionTok{pivot\_longer}\NormalTok{(}\AttributeTok{cols=}\FunctionTok{c}\NormalTok{(sens\_val, spec\_val),}
               \AttributeTok{names\_to =} \StringTok{"metric"}\NormalTok{,}
               \AttributeTok{values\_to =} \StringTok{"caret value"}\NormalTok{) }\SpecialCharTok{|\textgreater{}}
  \FunctionTok{left\_join}\NormalTok{(metrics\_table, }\AttributeTok{by=}\StringTok{"metric"}\NormalTok{) }\CommentTok{\#|\textgreater{}}
\end{Highlighting}
\end{Shaded}

\begin{verbatim}
## # A tibble: 2 x 3
##   metric   `caret value` value
##   <chr>            <dbl> <dbl>
## 1 sens_val         0.474    NA
## 2 spec_val         0.960    NA
\end{verbatim}

\begin{Shaded}
\begin{Highlighting}[]
 \CommentTok{\# knitr::kable(col.names = c("Metric", "Caret Values", "Function Values"))}
\end{Highlighting}
\end{Shaded}

The values from our functions and the values from the sensitivity and
specificity functions are the same.

Let's check the rest of our functions against the values from the
confusionMatrix function. Sensitivity and specificity can also be taken
from here.

\begin{Shaded}
\begin{Highlighting}[]
\NormalTok{caret\_acc }\OtherTok{\textless{}{-}}\NormalTok{ conf\_matrix}\SpecialCharTok{$}\NormalTok{overall[}\DecValTok{1}\NormalTok{]}
\NormalTok{caret\_err }\OtherTok{\textless{}{-}} \DecValTok{1}\SpecialCharTok{{-}}\NormalTok{caret\_acc}
\NormalTok{caret\_prec }\OtherTok{\textless{}{-}}\NormalTok{ conf\_matrix}\SpecialCharTok{$}\NormalTok{byClass[}\DecValTok{5}\NormalTok{]}
\NormalTok{caret\_sens }\OtherTok{\textless{}{-}}\NormalTok{ conf\_matrix}\SpecialCharTok{$}\NormalTok{byClass[}\DecValTok{1}\NormalTok{]}
\NormalTok{caret\_spec }\OtherTok{\textless{}{-}}\NormalTok{ conf\_matrix}\SpecialCharTok{$}\NormalTok{byClass[}\DecValTok{2}\NormalTok{]}
\NormalTok{caret\_f1 }\OtherTok{\textless{}{-}}\NormalTok{ conf\_matrix}\SpecialCharTok{$}\NormalTok{byClass[}\DecValTok{7}\NormalTok{]}

\NormalTok{caret\_results }\OtherTok{\textless{}{-}} \FunctionTok{as.data.frame}\NormalTok{(}\FunctionTok{cbind}\NormalTok{(caret\_acc, caret\_err, caret\_prec, caret\_sens, caret\_spec, caret\_f1))}
\FunctionTok{colnames}\NormalTok{(caret\_results) }\OtherTok{\textless{}{-}}\NormalTok{ metrics\_table}\SpecialCharTok{$}\NormalTok{metric}
\NormalTok{caret\_results }\SpecialCharTok{|\textgreater{}} 
  \FunctionTok{pivot\_longer}\NormalTok{(accuracy}\SpecialCharTok{:}\NormalTok{f1\_score,}
               \AttributeTok{names\_to=}\StringTok{"metric"}\NormalTok{,}
               \AttributeTok{values\_to=}\StringTok{"caret\_vals"}\NormalTok{) }\SpecialCharTok{|\textgreater{}}
  \FunctionTok{right\_join}\NormalTok{(metrics\_table) }\CommentTok{\#|\textgreater{}}
\end{Highlighting}
\end{Shaded}

\begin{verbatim}
## Joining with `by = join_by(metric)`
\end{verbatim}

\begin{verbatim}
## # A tibble: 6 x 3
##   metric           caret_vals value
##   <chr>                 <dbl> <dbl>
## 1 accuracy              0.807 0.807
## 2 error_rate            0.193 0.193
## 3 precision_rate        0.844 0.844
## 4 sensitivity_rate      0.474 0.474
## 5 specificity_rate      0.960 0.960
## 6 f1_score              0.607 0.607
\end{verbatim}

\begin{Shaded}
\begin{Highlighting}[]
  \CommentTok{\#knitr::kable(col.names = c("Metric", "Function Values", "Caret Values"))}
\end{Highlighting}
\end{Shaded}

The values for each of our functions are the same as the values for the
built in functions from the caret package.

\begin{enumerate}
\def\labelenumi{\arabic{enumi}.}
\setcounter{enumi}{12}
\tightlist
\item
  Investigate the pROC package. Use it to generate an ROC curve for the
  data set. How do the results compare with your own functions?
\end{enumerate}

\begin{Shaded}
\begin{Highlighting}[]
\FunctionTok{library}\NormalTok{(pROC)}
\end{Highlighting}
\end{Shaded}

\begin{verbatim}
## Type 'citation("pROC")' for a citation.
\end{verbatim}

\begin{verbatim}
## 
## Attaching package: 'pROC'
\end{verbatim}

\begin{verbatim}
## The following objects are masked from 'package:stats':
## 
##     cov, smooth, var
\end{verbatim}

\begin{Shaded}
\begin{Highlighting}[]
\FunctionTok{roc}\NormalTok{(class\_df}\SpecialCharTok{$}\NormalTok{class,}
\NormalTok{    class\_df}\SpecialCharTok{$}\NormalTok{scored.probability,}
    \AttributeTok{plot=}\NormalTok{T)}
\end{Highlighting}
\end{Shaded}

\begin{verbatim}
## Setting levels: control = 1, case = 0
\end{verbatim}

\begin{verbatim}
## Setting direction: controls > cases
\end{verbatim}

\includegraphics{homeworks2_metrics_files/figure-latex/unnamed-chunk-16-1.pdf}

\begin{verbatim}
## 
## Call:
## roc.default(response = class_df$class, predictor = class_df$scored.probability,     plot = T)
## 
## Data: class_df$scored.probability in 57 controls (class_df$class 1) > 124 cases (class_df$class 0).
## Area under the curve: 0.8503
\end{verbatim}

\end{document}
